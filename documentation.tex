\documentclass{article}
\usepackage[utf8]{inputenc}
\usepackage[T1]{fontenc}
\usepackage{lmodern}
\usepackage[polish]{babel}
\usepackage{amsmath}
\usepackage{tikz}
\usepackage{algorithm}
\usepackage{algpseudocode}
\usepackage{hyperref}
\usepackage{float}
\usepackage{graphicx}
\usepackage{mathtools}

\title{Internship JetBrains task documentation}
\author{Wojciech Typer}
\date{}

\begin{document}
\maketitle

\textbf{Effective file transfer between machines for remote development}

\section*{Overview}
This Rust program connects to a local HTTP server and continuously downloads data in chunks until it collects a user-specified number of bytes. After retrieving the full content, it computes and prints its SHA-256 hash.

\section*{Functionality}

\begin{itemize}
    \item Prompts the user for the total number of bytes to download.
    \item Connects to a server at \texttt{127.0.0.1:8080}.
    \item Sends HTTP \texttt{Range} requests to fetch data in 10\,000-byte segments.
    \item Extracts the HTTP response body from the received data.
    \item Appends the data to a buffer until the expected length is reached.
    \item Computes the SHA-256 hash of the resulting buffer.
    \item Prints the hash as a hexadecimal string.
\end{itemize}

\section*{Implementation Details}

\begin{itemize}
    \item The program uses \texttt{TcpStream} to send raw HTTP requests and receive responses.
    \item To extract the response body, it searches for the HTTP header terminator sequence \texttt{\textbackslash r\textbackslash n\textbackslash r\textbackslash n}.
    \item The hash is calculated using the \texttt{sha2} crate.
\end{itemize}
\end{document}
